\title{Introduction to Chemistry: Reactions and Ratios}


\documentclass[12pt]{article}

\begin{document}
\maketitle

\section{S5: Matter}

\begin{itemize}
\item Halogens (7A)
\item Noble Gases (8A)
\end{itemize}

Elements that exist as diatomic molecules
end with \textbf{"gens"} (e.g. hydrogen, halogen)

\section{S6 : Energy Part II}

% Measured in Joules\\
%\vspace{5mm}
%\noindent
% \\
 

\begin{itemize}
\item Measured in Joules
\item Tendency of universe towards \emph{lowest energy}
\end{itemize}

\begin{equation}
\delta E = E_f - E_i\footnote{f stands for final and i stands for initial}
\end{equation}

\section {S7 : Energy Part II}

\begin{itemize}
\item For \textbf{like-charged} particles, the energy \textbf{increases} the \textbf{closer} the particles are.
\item For \textbf{oppositely-charged} particles, the energy \textbf{decreases} the \textbf{closer} the particles are. 
\end{itemize}

Coulomb's Law:
\begin{equation}
\delta F = \frac{kq_1q_2}{\epsilon r^2}
\end{equation}

\section {S8 : Intro Stoichiometry Part I}

$HNO_2$ nitrous acid 

\section {S9 : Intro Stoichiometry Part II}

\textbf{Avogadro} constant:
\begin{equation}
6.022*10^{23} = 1\ mole
\end{equation}
The constant defined as the number of atoms in 12 grams of hydrogen of the isotope carbon-12

\end{document}
This is never printed